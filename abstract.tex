\documentclass{icsc}

\usepackage{booktabs}  % nice tables
\usepackage{natbib}
\usepackage{siunitx}  % use for all units
\usepackage{subcaption}  % for subfigures

\def\kph{\kilo\meter\per\hour}

\pagestyle{empty}

\begin{document}

\begin{center}
  \fontsize{14}{20}{\bf Balance Assist Bicycle Reduces Undesired Motions and
    Fall Probablity when Subject to Disturbances\\[3pt]}
\end{center}

%%%%%%%%%%%%%%%% authors %%%%%%%%%%%%%%%
\begin{center}
  \normalsize{\bf{J. K. Moore$^{*}$,
                  M. Haitjema$^{*}$,
                  L. Alizadehsaravi$^{*}$}}
\end{center}

\begin{center}
  \begin{tabular}{c}
    $^*$ Faculty of Mechanical Engineering\\
    Delft University of Technology\\
    Mekelweg, 2628CD Delft, The Netherlands\\
    e-mail: j.k.moore@tudelft.nl\\
  \end{tabular}
\end{center}

\begin{keywords}
  bicycle,
  fall,
  assist,
\end{keywords}

\section{INTRODUCTION}
%
We have developed a balance assisting bicycle that uses a custom electric
steering motor to apply a steering torque in parallel with the rider's steering
actions. The motor is controlled by a simple control algorithm based on
feedback from a rate gyroscope that measures roll angular rate. Applying a
rightward steering torque proportional to the rightward rolling rate  can lower
the speed range at which the bicycle feels self-stable to the rider. We
hypothesize that making the bicycle stable at lower speeds will reduce the
control action needed from the rider in basic balancing tasks and that use of
such a system may reduce falls. We have developed three comprehensive
experiments to assess the effectiveness of the balance assist effects. In the
first experiments we both distract the rider and apply small disturbances via
the steering motor during staight line balancing. In the second experiment, we
apply increasingly large perturbations to the handlebars via computer
controlled ropes while riding on a treadmill at low speeds. In the final
experiment, we perturb straight line riding via a kickplate, which slides the
ground out under the front wheel of the bicycle to see differences in recovery
response with and without balance assist.

\section{METHODS}
%
To design the controller we utlized the linear Carvallo-Whipple bicycle model.
The states are the roll angle \(\phi\) and steer angle \(\delta\) along with
their derivatives and the inputs are roll torque \(T_\phi\) and steer torque
\(T_\delta\). The state \(\mathbf{A}\) and input \(\mathbf{B} =
\begin{bmatrix}\mathbf{B}_\delta \quad \mathbf{B}_\delta\end{bmatrix} \)
matrices are functions of the equilbirum forward speed \(v\). The applied
steering torque \(T_\delta\) is the sum of the (h)uman and (m)otor torques,
with the motor torque following a proportional roll rate feedback law:
%
\begin{align}
  T_\delta =
  T_\delta^\textrm{h} + T_\delta^\textrm{m} =
  T_\delta^\textrm{h} - k_{\dot{\phi}}(v - v_\textrm{weave})\dot{\phi}
\end{align}
where \(v_\textrm{weave}\) is the open loop weave critical speed which gives
the model with the automatic control:
%
\begin{align}
  \dot{\vec{x}} =
  \left(
    \mathbf{A} - \mathbf{B}_\delta
    \left[0 \quad k_{\dot{\phi}}(v - v_\textrm{weave}) \quad 0 \quad 0\right]
  \right)
\vec{x} + \begin{bmatrix}\mathbf{B}_\delta \quad \mathbf{B}_\delta\end{bmatrix}
\begin{bmatrix} T_{\phi} \\ T_\delta^\textrm{h} \end{bmatrix}
\label{eq:state-space}
\end{align}

The gain \(k_{\dot{\phi}}\) can be selected such that the eigenvalues of
Equation~\ref{eq:state-space} have negative real parts for a range of speeds.

We developed a series of three experiments to assess whether stabilizing the
bicycle is beneficial in safety critical scenarios:
\begin{description}
  \item[Experiment 1] Comparison of 18 older and 14 younger adults in straight
    line riding when 1) asked to look over their shoulder and 2) subjected
    handlebar motor perturbations with the balance assist system on and off. We
    measured the standard deviation of their steer and roll motions after the
    disturbances.
  \item[Experiment 2] Twenty-six young adults were subjected to externally
    applied and measured handlebar torques of varying magnitudes while riding
    on a treadmill at low speeds 6~\si{\kph} and 10~\si{\kph} to determine the
    magintude of perturbation required to cause them to fall with the balance
    assist system on and off. We log the order of perturbations and their
    impulse as well as the bicycle state at the time of perturbation.
  \item[Experiment 3] We subject X particpants riding at 12~\si{\kph} to
    lateral pulsive perturbations at the front tire contact patch as they rode
    over a kickplate. We measured the standard deviation of their steer and
    roll motions after the disturbances.
\end{description}

\begin{figure}
  \begin{center}
    \subcaptionbox{Experiment 1}{\includegraphics[width=40mm]{example-image-a}}
    \subcaptionbox{Experiment 2}{\includegraphics[width=40mm]{example-image-b}}
    \subcaptionbox{Experiment 3}{\includegraphics[width=40mm]{example-image-c}}
    \label{fig:fig1}
  \end{center}
\end{figure}

We apply multivariate linear regression to the performance metrics derived from
each experiment (ANOVA Experiments 1 \& 3, logistic Experiment 2) and
investigate whether the balance assist system reduce response motions
(Experiments 1 \& 3) or if the probability of falling is reduced (Experiment
2).

\section{Results}

\thispagestyle{empty}

\section{CONCLUSIONS}
%
We very much look forward to welcoming you to Imabari, Japan! Best wishes and
the warmest regards from the Organizing Committee of International Cycling
Safety Conference 2024.

\begin{thebibliography}{6}
% \begin{thebibliography}{66} % use this for more than 9 references
  \bibitem{Pac02} H.~B.~Pacejka, \textit{Tyre and Vehicle Dynamics},
    Butterworth and Heinemann, Oxford, 2002.
  \bibitem{Ber07} E.~Bertolazzi, F.~Biral, M.~Da~Lio and V.~Cossalter, ``The
    influence of rider's upper body motions on motorcycle minimum time
    maneuvering'', in C.~L.~Bottasso, P.~Masarati and L.~Trainelli (eds),
    \textit{Proceedings, Multibody Dynamics 2007, ECCOMAS Thematic Conference},
    Milano, Italy, 25--28 June 2007, Politecnico di Milano, Milano, 2007,
    15~pp.
  \bibitem{Sha71} R.~S.~Sharp, ``The stability and control of motorcycles'',
    \textit{Proceedings of the IMechE, Part C, Journal of Mechanical
    Engineering Science} \textbf{13} (1971), pp.~316--329.
\end{thebibliography}
\thispagestyle{empty}
\end{document}
