\documentclass{icsc}

\pagestyle{empty}

\begin{document}

\begin{center}
  \fontsize{14}{20}{\bf Balance Assist Bicycle Reduces Undesired Motions and
  Reduces Fall Probablity when Subject to Perturbations\\[3pt]}
\end{center}

%%%%%%%%%%%%%%%% authors %%%%%%%%%%%%%%%
\begin{center}
  \normalsize{\bf{J. K. Moore$^{*}$,
                  M. Haitjema$^{*}$,
                  L. Alizadehsaravi$^{*}$}}
\end{center}

\begin{center}
  \begin{tabular}{c}
    $^*$ Faculty of Mechanical Engineering\\
    Delft University of Technology\\
    Mekelweg, 2628CD Delft, The Netherlands\\
    e-mail: j.k.moore@tudelft.nl\\
  \end{tabular}
\end{center}

\begin{keywords}
  bicycle,
  fall,
  assist,
\end{keywords}

\section{INTRODUCTION}
%
We have developed a balance assisting bicycle that uses a custom electric
steering motor to apply a steering torque in parallel with the rider's steering
actions. The motor is controlled by a simple control algorithm based on
feedback from a rate gyroscope that measures roll angular rate. Applying a
rightward steering torque proportional to the rightward rolling rate  can lower
the speed range at which the bicycle feels self-stable to the rider. We
hypothesize that making the bicycle stable at lower speeds will reduce the
control action needed from the rider in basic balancing tasks and that use of
such a system may reduce falls. We have developed three comprehensive
experiments to assess the effectiveness of the balance assist effects. In the
first experiments we both distract the rider and apply small disturbances via
the steering motor during staight line balancing. In the second experiment, we
apply increasingly large perturbations to the handlebars via computer
controlled ropes while riding on a treadmill at low speeds. In the final
experiment, we perturb straight line riding via a kickplate, which slides the
ground out under the front wheel of the bicycle to see differences in recovery
response with and without balance assist.

\section{GENERAL INSTRUCTIONS}
%
The abstract must be written in English. It must contain the name, address and
e-mail address of each author. The abstract should be no longer than three
pages including references. All page margins should be 1''. The paper used
should have the size 210\,x\,297\, mm which is the European (German) A4 size.
It is suggested to use styles for formatting and automatic reference, figure
numbering to avoid editorial errors. To avoid compatibility problems it is
advised to use only upper or lower case Latin alphabet, numbers and the
underscore character in the file name.

\subsection{Equations}
%
Equations should be numbered continuously according to the format shown in
Equation~(\ref{eq:equ1}):
\begin{equation} \label{eq:equ1}
  e^{i\pi} + 1 = 0,
\end{equation}

where the unknown symbols are explained after the equation.

\subsection{Figures and tables}
%
\begin{table}[h!]
  \begin{center}
    \caption{Example of a table with a short caption.} \label{tab:tab1}
    \begin{tabular}{|c|ccc|}
      \hline
      &  $x$  &  $y$  &  $z$ \\
      \hline
      $x'$  &  $\alpha_1$ & $\beta_1$ & $\gamma_1$ \\
      $y'$  &  $\alpha_2$ & $\beta_2$ & $\gamma_2$ \\
      $z'$  &  $\alpha_3$ & $\beta_3$ & $\gamma_3$ \\
      \hline
    \end{tabular}
  \end{center}
\end{table}
All figures should be clearly readable and relevant to the presented text. Use
of at least 300\,dpi resolution for pictures and 600\,dpi for line art is
required, 1\,px wide lines in figures should be avoided as they may become
invisible in print. There is no limit on the amount of figures as long as they
do not dominate the text and the total length of the paper is within the
specified limits. Both figures and tables should be centred on the page.

\begin{figure}[h!]
\begin{center}
  \includegraphics[width=55mm]{figure1}
  \caption{An example of a figure caption. Use 10~pt Times New Roman.
           For long captions, use a text width of 13~cm.
           Use the same style for the tables.} \label{fig:fig1}
\end{center}
\end{figure}
Figures, graphs and tables must be included in the same style as shown for
Figure~\ref{fig:fig1} and Table~\ref{tab:tab1}.

\subsection{References}
%
Bibliographical citations should be written in the order in which they are
cited, see the References section below, where Reference~\cite{Pac02}
exemplifies the case of a textbook, while Reference~\cite{Ber07} is an article
in conference proceedings and Reference~\cite{Sha71} is an article in a
journal. Use can be made of a pre-existing bibtex style which uses a similar
style.

\thispagestyle{empty}

\section{PROCEEDINGS}
%
The final version of your extended abstract will be part of the proceedings.

\section{CONCLUSIONS}
%
We very much look forward to welcoming you to Imabari, Japan! Best wishes and
the warmest regards from the Organizing Committee of International Cycling
Safety Conference 2024.

\begin{thebibliography}{6}
% \begin{thebibliography}{66} % use this for more than 9 references
  \bibitem{Pac02} H.~B.~Pacejka, \textit{Tyre and Vehicle Dynamics},
    Butterworth and Heinemann, Oxford, 2002.
  \bibitem{Ber07} E.~Bertolazzi, F.~Biral, M.~Da~Lio and V.~Cossalter, ``The
    influence of rider's upper body motions on motorcycle minimum time
    maneuvering'', in C.~L.~Bottasso, P.~Masarati and L.~Trainelli (eds),
    \textit{Proceedings, Multibody Dynamics 2007, ECCOMAS Thematic Conference},
    Milano, Italy, 25--28 June 2007, Politecnico di Milano, Milano, 2007,
    15~pp.
  \bibitem{Sha71} R.~S.~Sharp, ``The stability and control of motorcycles'',
    \textit{Proceedings of the IMechE, Part C, Journal of Mechanical
    Engineering Science} \textbf{13} (1971), pp.~316--329.
\end{thebibliography}
\thispagestyle{empty}
\end{document}
